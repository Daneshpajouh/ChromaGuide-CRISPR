\chapter{Research Questions and Objectives}

This proposal is organized around three coupled modules: \textbf{(i) on-target efficacy prediction}, \textbf{(ii) off-target risk prediction}, and \textbf{(iii) integrated \sgRNA\ design}. The research questions and hypotheses below are stated so they can be evaluated on leakage-controlled and shift-aware splits.

\section{Research questions and objectives}

\textbf{RQ1 (on-target module):} \emph{How much does adding epigenomic context and multi-modal fusion improve on-target efficacy prediction under realistic generalization constraints (e.g., gene-held-out and dataset/cell-line shift)?}

\textbf{Objective 1 (on-target modeling):} Develop a leakage-evaluated on-target predictor that fuses sequence with epigenomic context (and optional \gRNA{}-structure features) using a fusion mechanism that encourages complementary (non-redundant) representations.

\textbf{RQ2 (off-target module):} \emph{Can an off-target model that uses mismatch/bulge patterns and (when available) local chromatin context produce better guide-level specificity/risk estimates than sequence-only baselines?}

\textbf{Objective 2 (off-target modeling):} Build an off-target module that (i) enumerates plausible unintended sites and (ii) scores each site to yield an aggregated guide-level off-target risk/specificity estimate.

\textbf{RQ3 (integrated \sgRNA\ design):} \emph{Does combining calibrated on-target predictions with aggregated off-target risk improve end-to-end guide ranking for practical \sgRNA\ design compared to optimizing either objective alone?}

\textbf{Objective 3 (end-to-end guide design):} Define and validate an integrated \sgRNA\ design score that jointly ranks candidates by predicted efficiency and specificity, with explicit, tunable trade-offs.

\textbf{Cross-cutting objective (uncertainty and shift):} Provide distribution-free prediction intervals for on-target outcomes via split conformal prediction \emph{under the exchangeability assumption} (including weighted variants for dataset/cell-line shift, which rely on (estimated) importance weights / likelihood ratios)~\citep{vovk2005,romano2019,barber2023}, and evaluate transfer learning under low-resource and shifted settings~\citep{pan2010,liao2025primenet,mathis2024pridict2}.

\section{Testable hypotheses}

\begin{hypothesis}[On-target generalization gain]
\label{hyp:h1}
On leakage-controlled gene-held-out splits, adding epigenomic context and multi-modal fusion improves on-target predictive performance (e.g., Spearman correlation) over a strong sequence-only baseline.
\end{hypothesis}

\textbf{Operationalization:} On a pre-registered, leakage-controlled gene-held-out test split (Split~A), ChromaGuide improves Spearman correlation over the chosen baseline by at least $\Delta\rho = 0.004$ (e.g., $0.876\to 0.880$) while maintaining conformal prediction-interval coverage within $\pm 0.02$ of the target level (e.g., 0.90). (ChromeCRISPR's reported $\rho=0.876$ was obtained under a random 85/15 split; we will re-evaluate the baseline under Split~A for like-for-like comparison.)

\begin{hypothesis}[Off-target risk estimation gain]
\label{hyp:h2}
Conditioned on the same candidate guides, incorporating chromatin context (when available) improves per-site off-target scoring and the resulting aggregated guide-level off-target risk/specificity estimate compared to sequence-only scoring.
\end{hypothesis}

\textbf{Operationalization:} On an off-target benchmark with measured unintended sites, the chromatin-aware model improves a ranking metric (e.g., AUROC/AUPRC for site classification or NDCG/Top-$k$ recall for site ranking) over the sequence-only baseline, and this improvement persists under cell-line-held-out evaluation when epigenomic tracks are available.

\begin{hypothesis}[Integrated design improves guide ranking]
\label{hyp:h3}
A joint \sgRNA\ design score that combines calibrated on-target predictions with aggregated off-target risk yields better guide ranking than optimizing on-target alone or off-target alone.
\end{hypothesis}

\textbf{Operationalization:} When ranking candidate guides per target (gene/region), the integrated score improves Top-$k$ selection quality (e.g., probability that at least one of the top-$k$ guides is high-efficacy and low-risk) relative to (i) on-target-only ranking and (ii) off-target-only ranking, under the same evaluation splits.

\section{Expected contributions}

\begin{itemize}
\item \textbf{Multi-modal on-target efficacy prediction:} a leakage-evaluated model that fuses sequence, optional \gRNA\ structure, and ENCODE-style epigenomic context features to predict on-target editing outcomes, with optional calibrated uncertainty.
\item \textbf{Chromatin-aware off-target risk prediction:} an off-target module that enumerates plausible unintended sites and scores them (mismatch/bulge patterns and, when available, local chromatin context) to produce aggregated guide-level specificity/risk estimates.
\item \textbf{End-to-end \sgRNA\ design score:} a practical ranking score that explicitly balances predicted on-target efficiency against aggregated off-target risk to support complete guide RNA design.
\item \textbf{Evaluation protocol for biological shift:} a set of leakage-controlled gene-held-out and dataset/cell-line-held-out splits (and stress tests) that make generalization and uncertainty failure modes visible.
\end{itemize}

\section{Chapter roadmap}

Chapter~1 introduces CRISPR-Cas9 guide design and motivates the need for context-aware, uncertainty-aware prediction. Chapter~2 (this chapter) states the research questions, hypotheses, and expected contributions. Chapter~3 reviews relevant literature (CRISPR biology, epigenomics, existing predictors, uncertainty, and transfer learning). Chapter~4 describes the proposed methodology for the on-target, off-target, and integrated design modules. Chapter~5 presents the evaluation plan and expected results, emphasizing leakage-controlled gene-held-out tests and cross-domain (dataset/cell-line) stress tests. Chapter~6 provides the project timeline, and Chapter~7 discusses broader impacts.
