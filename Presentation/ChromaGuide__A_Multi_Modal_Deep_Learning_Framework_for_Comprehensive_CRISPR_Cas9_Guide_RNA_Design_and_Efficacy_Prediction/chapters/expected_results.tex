\chapter{Expected Results and Contributions}

This chapter summarizes expected, \emph{testable} outcomes under the leakage-controlled protocol (Split A primary; Splits B/C stress tests).

\section{Expected results (empirical)}

\begin{sloppypar}
\begin{itemize}
\item \textbf{Primary ranking performance:} On Split~A (gene-held-out), achieve at least $\Delta\rho\geq 0.004$ improvement over the ChromeCRISPR baseline when both are evaluated on the \emph{same} leakage-controlled splits. (ChromeCRISPR's reported $\rho=0.876$ was obtained under a random 85/15 split; we will re-evaluate the baseline under Split~A for like-for-like comparison.) See~\citep{daneshpajouh2025chromecrispr}.
\item \textbf{Reliable uncertainty under shift:} Achieve empirical interval coverage within $\pm 0.02$ of the target level (e.g., 0.90) \emph{under the exchangeability assumption} and report coverage/width stratified by dataset and cell line (Splits B/C).
\end{itemize}
\end{sloppypar}


\section{Expected contributions}\label{sec:expected-contributions}

This work targets publishable contributions by combining (i) multi-modal modeling and uncertainty quantification under biological shift with (ii) leakage-controlled, reproducible evaluation and tooling for chromatin-aware \gRNA\ scoring.

\subsection{Contributions to Computer Science and Machine Learning}

\begin{itemize}
\item \textbf{Multi-modal fusion with an explicit non-redundancy objective (MINE/CLUB).} We will develop and ablate a fusion mechanism that is expressive (e.g., gating/attention/MLP fusion) while explicitly discouraging redundant sequence--epigenomic representations~\citep{belghazi2018,cheng2020}.
\item \textbf{Shift-aware uncertainty via weighted conformal prediction.} We will adapt and evaluate \emph{weighted} conformal calibration as a practical uncertainty layer for genomic predictors under covariate shift and batch effects. Since weighted conformal methods rely on (estimated) importance weights / likelihood ratios, we will clearly report where coverage holds and where it degrades.
\item \textbf{Bounded-outcome regression with distribution-free intervals.} We pair a bounded head (Beta regression for $y\in(0,1)$) with split/weighted conformal prediction to produce calibrated intervals with coverage \emph{under the exchangeability assumption} that remain meaningful near the boundaries.
\item \textbf{Shift-focused benchmark protocol.} We provide standardized evaluation across gene-held-out, dataset-held-out, and cell-line-held-out splits with consistent reporting of ranking and interval metrics.
\end{itemize}

\subsection{Contributions to Computational Biology}

\begin{itemize}
\item \textbf{Context-aware CRISPR on-target prediction with calibrated uncertainty.} ChromaGuide integrates chromatin accessibility/state with explicit empirical coverage targets to quantify when context-aware predictions are reliable in a given cell context.
\item \textbf{Missing-modality robustness via modality dropout.} We treat uneven epigenomic availability as a first-class setting and report sequence-only vs. multi-modal behavior under controlled masking.
\item \textbf{Transfer protocol (optional; ablated).} We evaluate initialization/fine-tuning from strong sequence-only encoders to context-aware models when context features become available.
\item \textbf{Reusable pipeline for chromatin-aware \gRNA\ scoring.} We deliver an end-to-end workflow for context feature extraction, multi-modal training, and calibrated inference.
\end{itemize}

\subsection{Contributions to Bioinformatics}

\begin{itemize}
\item \textbf{Leakage-controlled split construction.} We specify gene-held-out, dataset-held-out, and cell-line-held-out splits with deduplication and tuning restrictions to reduce optimistic bias.
\item \textbf{Auditable preprocessing for heterogeneous screens.} We standardize sequence extraction, coordinate handling, outcome scaling/clipping for bounded regression, and training-only normalization for epigenomic tracks.
\item \textbf{Reproducibility artifacts.} We package the pipeline as reproducible workflows/checkpoints/configs to enable re-running ablations and comparisons.
\item \textbf{Benchmark data products.} We provide processed tables (or scripts to generate them) with explicit split assignments and metadata for fair comparison.
\end{itemize}

\subsection{Contributions to Biology and Genomics}

\begin{itemize}
\item \textbf{Quantifying context value beyond sequence.} Through controlled ablations, we estimate the incremental contribution of accessibility/histone-mark context over sequence-only baselines across genes and cell contexts.
\item \textbf{Interpretability for mechanistic hypotheses.} We report sequence- and context-level attributions (e.g., PAM/seed sensitivity; accessibility/histone-mark saliency) to support qualitative error analysis and hypothesis generation.
\item \textbf{Decision support with intervals.} The model outputs a score and a calibrated interval to support risk-aware guide selection and explicit performance--reliability trade-offs.
\item \textbf{When context matters most.} Under cell-line-held-out and dataset-held-out tests, we characterize regimes where chromatin context yields larger gains versus regimes where sequence dominates.
\end{itemize}

\section{Responsible use and limitations (brief)}

\begin{itemize}
\item \textbf{Data coverage:} When matched epigenomic tracks are unavailable, results are reported with and without the affected contexts; missingness robustness is treated explicitly.
\item \textbf{Calibration limits under batch effects:} Weighted conformal methods can fail under severe non-exchangeability; coverage is audited across held-out domains and failures are analyzed.
\item \textbf{Scope and governance:} Outputs are intended for \gRNA\ prioritization and hypothesis generation (not clinical decisions); releases include limitations, validation guidance, and dual-use-aware documentation.
\end{itemize}
