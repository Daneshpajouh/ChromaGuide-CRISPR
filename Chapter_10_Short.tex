% ======================================================================
% CHAPTER 10: GENERALIZATION AND DOMAIN SHIFT (SHORT VERSION)
% ======================================================================

\chapter{Cross-Dataset Generalization and Domain Analysis}

A critical challenge in computational biology is generalizing from training cell types (e.g., K562) to novel patient tissues (e.g., Liver). This chapter quantifies this "Domain Shift" and provides strategies to overcome it.

\section{Quantifying Domain Shift}
We use \textbf{Maximum Mean Discrepancy (MMD)} to measure the statistical distance between cell types. We observe a strong negative linear relationship between MMD and model performance ($R^2 = 0.91$).

\begin{table}[H]
\centering
\caption{Performance Degradation by Domain Distance}
\begin{tabular}{|l|c|c|}
\hline
\textbf{Training $\to$ Test} & \textbf{MMD (Distance)} & \textbf{Spearman $\rho$} \\
\hline
K562 $\to$ K562 & 0.00 & 0.97 \\
\hline
K562 $\to$ HEK293 & 0.19 & 0.95 \\
\hline
K562 $\to$ Hepatocytes & 0.46 & 0.87 \\
\hline
\end{tabular}
\end{table}

Feature analysis reveals that \textbf{ATAC-seq (chromatin accessibility)} accounts for 41\% of this shift. The model effectively learns "cell-type grammar," so when the grammar changes (different accessibility rules), performance degrades predictably.

\section{Strategies for New Tissues}
Because the degradation is predictable, we can effectively mitigate it:

\begin{enumerate}
    \item \textbf{Fine-Tuning:} Just 100 experimental data points in a new tissue are sufficient to recover near-optimal performance ($\rho$ rises from 0.87 to 0.94).
    \item \textbf{Prediction:} We can use the MMD score to \textit{predict} how well the model will work on a new patient sample before running any experiments, serving as a reliability index.
\end{enumerate}

This framework allows safe deployment to novel biological contexts by flagging when local validation is required.
