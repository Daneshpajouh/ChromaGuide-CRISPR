\chapter{Timeline and Work Plan}

\section[Timeline and Milestones]{Timeline and Milestones}

\textbf{Scope.} Accelerated 42
execution window (February 2026--December 2026). Dates below are stated explicitly to avoid ambiguity.

\subsection{February 2026: Proposal and scoping}

\textbf{Milestone 1 (by February 26, 2026): Finalized proposal + execution checklist}
\begin{itemize}
\item Deliverable: finalized evaluation protocol and leakage-controlled splits specification (Split A primary; Splits B/C optional stress tests)
\item Deliverable: concrete implementation plan (data, preprocessing, model, calibration, reporting) and compute budget
\end{itemize}

\subsection{March--April 2026: Implementation and baseline reproduction}

\textbf{Milestone 2 (by April 30, 2026): Working training/evaluation pipeline + baselines}
\begin{itemize}
\item Deliverable: reproducible pipeline for preprocessing, splitting, training, and evaluation
\item Deliverable: baseline reproduction on the same splits (ChromeCRISPR~\citep{daneshpajouh2025chromecrispr}, DeepSpCas9~\citep{kim2019deepspcas9}; CRISPRon~\citep{xiang2021} where comparable inputs exist)
\item Deliverable: first ChromaGuide prototype (sequence encoder + epigenomic feature extraction + simple fusion)
\end{itemize}

\subsection{May--July 2026: Final experiments and ablation analysis}

\textbf{Milestone 3 (by July 31, 2026): Completed experiments and ablation results}
\begin{itemize}
\item Deliverable: final experiments and ablations (epigenomics, fusion choice, calibration), reported on the primary gene-held-out test split
\item Deliverable: uncertainty reporting (conformal prediction intervals; empirical coverage and interval width)
\end{itemize}

\subsection{August--September 2026: Thesis writing and results integration}

\textbf{Milestone 3b (by September 30, 2026): Thesis-ready results package}
\begin{itemize}
  \item Deliverable: thesis chapters drafted---methods, evaluation protocol, results, and limitations
  \item Deliverable: all figures, tables, and supplementary material assembled
\end{itemize}

\subsection{October 2026--December 2026: Defense preparation and submission}

\textbf{Milestone 4 (by December 11, 2026): Defense-ready thesis and presentation}
\begin{itemize}
\item Deliverable: full thesis polish, formatting, and final submission package
\item Deliverable: defense slides + practice talk(s) + committee feedback incorporated
\end{itemize}

% Timeline figure omitted for concision.

\section{Computational Resources}

\textbf{HPC Cluster:} Digital Research Alliance Canada (NVIDIA V100/A100 GPUs)

\textbf{Training Time:} $\sim$20 seconds per iteration on V100 (estimated 100--200 hours total)

\textbf{Memory:} 32GB GPU memory per node

% Table omitted for concision.

\section{Data Resources}

\begin{itemize}

\item \textbf{ENCODE:} Public epigenomic tracks (100+ cell types), or similar resources providing accessibility and histone-mark signals.

\item \textbf{DeepHF/CRISPRon/CRISPR-FMC:} Public datasets (and related datasets), subject to compatibility with the proposed inputs and evaluation protocol.

\item \textbf{PRIDICT/PrimeNet:} Public models and datasets (or comparable alternatives) that may be used for initialization or transfer studies, pending feasibility analysis.

\item \textbf{Supplementary sources (as needed):} The above may be supplemented with additional public screens, cell-line-specific epigenomic tracks, or curated subsets when required for robustness analyses.

\end{itemize}

\section{Risk Assessment and Mitigation}

\begingroup
\let\oldsection\section
\let\oldsubsection\subsection
\renewcommand{\section}{\oldsubsection}
\renewcommand{\subsection}{\oldsubsubsection}

\section{Key risks and mitigations}

\begin{description}
\item[Epigenomic coverage (missing tracks).] Use modality dropout and/or hierarchical imputation; report results with and without imputed contexts, and treat missing-modality robustness as a primary analysis dimension.
\item[Transfer learning adds no value.] Keep transfer as an optional, fully ablated extension; gate it by a simple similarity/shift analysis and prioritize a strong in-domain baseline.
\item[Coverage violations under batch effects.] Use weighted conformal calibration and report coverage stratified by dataset/cell line; when coverage degrades, include failure-mode analyses rather than aggregating away the issue.
\item[Performance plateau.] Focus contributions on (i) leakage-controlled evaluation, (ii) context integration, and (iii) calibrated uncertainty; ensure publishable outcomes even when ranking gains are modest.
\end{description}

\section{Success criteria (operational)}

\begin{itemize}
\item \textbf{Minimum:} Match ChromeCRISPR's reported Spearman (0.876; reported under a random 85/15 split) and re-evaluate the ChromeCRISPR baseline on our leakage-controlled Split~A for a fair like-for-like comparison~\citep{daneshpajouh2025chromecrispr}, while producing calibrated prediction intervals with audited empirical coverage.
\item \textbf{Target:} Achieve 0.88--0.92 Spearman on Split A with empirical coverage within $\pm 0.02$ of the target level (e.g., 0.90), under the February--March 2026 scope.
\end{itemize}

\endgroup