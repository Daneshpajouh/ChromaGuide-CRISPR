\chapter{Proposed Validation and Anticipated Results}

% ======================================================================
% CHAPTER 4: EXPERIMENTAL VALIDATION (SHORT VERSION)
% ======================================================================

\section{Proposed Experimental Validation Strategy}

To validate the theoretical advantages of CRISPRO-MAMBA-X, we propose a rigorous two-phase validation campaign combining large-scale computational benchmarking with retrospective validation on high-quality external datasets.

\subsection{Phase 1: Computational Benchmarking (Anticipated)}
We will benchmark the model against current SOTA architectures (DeepHF \cite{DeepHF2019}, CRISPR-FMC \cite{Li2025}, Transformer \cite{Transformer2017}) on independent datasets (n=60,000).

\textbf{Hypothesis:} We hypothesize that integrating 1.2 Mbp context will significantly reduce the variance gap. Based on preliminary architectural simulations, we target the following performance metrics:

\begin{table}[H]
\centering
\caption{Target Performance Specifications (Spearman Correlation)}
\begin{tabular}{|l|c|c|c|}
\hline
\textbf{Model} & \textbf{DeepHF Test} & \textbf{ESP Dataset} & \textbf{Improvement (Target)} \\
\hline
CNN Baseline & 0.71 & 0.68 & - \\
\hline
Transformer & 0.84 & 0.79 & +18\% \\
\hline
\textbf{CRISPRO-MAMBA-X} & \textbf{$>0.92$} & \textbf{$>0.90$} & \textbf{$>30\%$} \\
\hline
\end{tabular}
\end{table}

A target correlation of $\rho > 0.95$ would represent a transformative leap in predictive reliability, effectively solving the "Context Blindness" problem.

\subsection{Simulated Ablation Analysis}
We expect the performance gains to be driven by two distinct factors. Our theoretical model suggests:
\begin{enumerate}
    \item \textbf{Epigenomic Integration:} ~12\% gain from capturing physical accessibility.
    \item \textbf{Long-Context Mamba:} ~14\% gain from capturing distal regulatory elements (TADs).
\end{enumerate}

\begin{figure}[h!]
    \centering
    \fbox{\parbox{0.9\textwidth}{\centering \vspace{1cm} \textbf{FIGURE PLACEHOLDER} \\ \textbf{File:} figures/fig\_9\_5.png \\ \textbf{Description:} Anticipated Source of Performance Gains. We expect Epigenomics and Long-Context Mamba to contribute roughly equally to bridging the variance gap. \vspace{1cm}}}
    \caption[Anticipated Ablation]{Anticipated Source of Performance Gains. We expect Epigenomics and Long-Context Mamba to contribute roughly equally to bridging the variance gap.}
    \label{fig:ablation_short}
\end{figure}

\subsection{Phase 2: Retrospective Validation on External Cohorts}
We will validate the model's physical precision using gold-standard off-target datasets from the literature (e.g., \textbf{CRISPRoffT} (2025) \cite{Du2025}, \textbf{BreakTag} (2024) \cite{Longo2024}).
\textbf{Success Metric:} A correlation of $\rho > 0.90$ between predicted off-target scores and the experimentally measured cleavage rates in these independent studies (n > 8,000 sites) would confirm that the model's "Physical Accessibility Gate" reflects biological reality, without requiring new wet-lab experiments.

% ======================================================================
% CHAPTER 10: GENERALIZATION (SHORT VERSION)
% ======================================================================

\section{Anticipated Generalization and Domain Analysis}

A critical risk for any ML model is domain shift. We plan to quantify this using Maximum Mean Discrepancy (MMD).

\subsection{Predicted Domain Shift Impact}
We anticipate a linear degradation in performance as the MMD distance between training and test cell types increases.

\begin{table}[H]
\centering
\caption{Hypothesized Performance vs. Domain Distance}
\begin{tabular}{|l|c|c|}
\hline
\textbf{Training $\to$ Test} & \textbf{MMD (Distance)} & \textbf{Expected $\rho$} \\
\hline
K562 $\to$ K562 & 0.00 & 0.97 \\
\hline
K562 $\to$ HEK293 & 0.19 & 0.95 \\
\hline
K562 $\to$ Hepatocytes & 0.46 & 0.87 \\
\hline
\end{tabular}
\end{table}

\subsection{Mitigation Strategy}
We propose a **Fine-Tuning Protocol** where small sample sizes (n=100) from the target tissue are used to recalibrate the model. We expect this to recover performance to $\rho > 0.94$, ensuring safe clinical deployment even in novel tissues.
