\chapter{Introduction}

\section{Motivation and problem statement}

CRISPR-Cas9 is a programmable RNA-guided nuclease that enables targeted genome editing in eukaryotic cells~\citep{jinek2012,cong2013,hsu2014}. In a standard SpCas9 workflow, a \gRNA{} directs \CasNine\ to a locus adjacent to a \PAM{} (5$'$-NGG-3$'$) on the non-target strand (NTS); recognition occurs via Arg1333 and Arg1335 major-groove contacts with the GG dinucleotide~\citep{jinek2012,anders2014}, while the target-strand complement (5$'$-CCN-3$'$) is not directly read. Downstream double-strand break (DSB) repair then produces an observed editing outcome.

This proposal targets \emph{comprehensive CRISPR-Cas9 guide RNA design and efficacy prediction}: given a candidate \gRNA\ and its genomic target, (i)~estimate on-target cleavage/indel efficiency in the \emph{relevant} cellular context, (ii)~quantify genome-wide off-target risk, and (iii)~produce an integrated design score that balances efficacy and safety. Existing predictors can achieve strong benchmark correlations, but practical deployment remains limited by two recurring issues: (i) models are often sequence-centric and only weakly capture chromatin accessibility/state, and (ii) most models output point estimates without calibrated uncertainty, making it hard to decide when predictions are reliable under dataset/cell-line shift~\citep{horlbeck2016,isaac2016,vovk2005,romano2019,barber2023}.


\section{Proposed approach and thesis overview}

Recent large-scale benchmarking work highlights that strong within-dataset performance does not always translate to reliable generalization across datasets and cell types~\citep{konstantakos2022,chen2022evaluation}.  Our group's prior work established a clear research progression: Daneshpajouh, Fowler, and Wiese~\citep{daneshpajouh2023navitas,daneshpajouh2023comparison} first conducted a systematic comparison of machine learning models for CRISPR/Cas on-target prediction (IEEE CIBCB 2023, Canadian AI 2023), demonstrating that RNNs significantly outperform CNNs and that GC content improves predictions. Building on these findings, ChromeCRISPR (Daneshpajouh, Fowler, and Wiese)~\citep{daneshpajouh2025chromecrispr}---a CNN-RNN hybrid trained on $\sim$60,000 sgRNAs from the DeepHF dataset---achieved state-of-the-art Spearman $\rho=0.876$ (MSE 0.0093), outperforming both DeepHF ($\rho=0.867$) and AttCRISPR ($\rho=0.872$). ChromaGuide extends this validated hybrid architecture by integrating epigenomic context, calibrated uncertainty via conformal prediction, and off-target risk assessment.

Motivated by these findings, we position ChromaGuide against both established sequence-based baselines and recent 2025 deep-learning models, including CRISPR\_HNN~\citep{li2025crisprhnn}, PLM-CRISPR~\citep{hou2025plmcrispr}, CRISPR-FMC~\citep{li2025crisprfmc}, DNABERT-Epi~\citep{kimata2025dnabertepi},  and Graph-CRISPR~\citep{jiang2025graph-crispr}, using the ChromeCRISPR benchmark protocols where applicable~\citep{daneshpajouh2025chromecrispr}.

\textbf{ChromaGuide} is a three-module \sgRNA{} prioritization pipeline:
\begin{enumerate}
    \item \textbf{On-target module:} multi-modal efficacy prediction with uncertainty quantification.
    \item \textbf{Off-target module:} genome-wide off-target site scoring.
    \item \textbf{Design module:} integrated score balancing efficacy and safety.
\end{enumerate}
Operational success is defined by \emph{surpassing current state-of-the-art models} through improved ranking performance on the primary gene-held-out test split (target Spearman $\geq 0.88$ with $\Delta\rho\geq 0.02$ vs.~the ChromeCRISPR baseline when evaluated on the same leakage-controlled split; note that ChromeCRISPR's baseline $\rho=0.876$ was obtained under a different split protocol---see Section~\ref{sec:evaluation} for details) and empirical interval coverage within $\pm 0.02$ of the target level (e.g., 0.90) \emph{under the exchangeability assumption}.

Chapter~\ref{ch:rq} formalizes the research questions, testable hypotheses, operational success criteria, and expected contributions of this work. Chapter~\ref{ch:background} reviews the relevant literature, and Chapter~\ref{ch:methodology} details the proposed methodology.